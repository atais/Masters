\documentclass[twoside,12pt]{report}
\usepackage[T1]{fontenc}
\usepackage[utf8]{inputenc}
%\usepackage[polish]{babel}
\usepackage{graphicx}
\usepackage{pdfpages}
\usepackage{hyphenat}
\usepackage{amsmath,amssymb,amsfonts}
\usepackage{txfonts}
\usepackage{geometry}
\usepackage{hyperref}
\usepackage{listings}
\usepackage{fancyhdr}
\usepackage{indentfirst}

%\selectlanguage{polish}


\geometry{a4paper,left=35mm,right=25mm,top=25mm,bottom=25mm}

\renewcommand{\headrulewidth}{0.1pt}
\renewcommand{\chaptername}{Rozdział}
\renewcommand{\contentsname}{Spis treści}
\renewcommand{\figurename}{Rys.}
\renewcommand{\tablename}{Tab.}
\renewcommand{\lstlistingname}{Listing}
\renewcommand{\listfigurename}{Spis rysunków}
\renewcommand{\listtablename}{Spis tabel}
\renewcommand{\lstlistlistingname}{Spis listingów}
\renewcommand{\bibname}{Bibliografia}

\pagestyle{fancy}
\fancyhf{}
\fancyhead[LE,RO]{\rightmark}
\fancyfoot[LE,RO]{\thepage}

\newtheorem{definition}{Definicja} % przykład nowego środowiska 
\newtheorem{example}{Przykład}[chapter] % przykład nowego środowiska 
\newtheorem{corollary}{Wniosek}[chapter] % przykład nowego środowiska 

\begin{document}

\includepdf[pages={1,2}]{img/title-page.pdf}

\tableofcontents	% generuje spis treści ze stronami !!!

\chapter{Wstęp} \label{rozdz.wstep} 

\section{Problematyka i zakres pracy}

Niniejsza praca obejmuje zagadnienia z zakresu inżynierii oprogramowania i sztucznej inteligencji. Głównym jej celem jest stworzenie aplikacji optymalizującej strukturę sieci drogowej.\\

Problemy komunikacji w dzisiejszych miastach są wszystkim znane. Zatory drogowe i korki w godzinach szczytu są chlebem powszednim. Pomimo wielu prób i sposobów, wciąż nie istnieje metoda jednoznacznie rozwiązująca tę kwestię. Bezspornie, dotyczy to wszystkich miast na świecie. Z teoretycznego punktu widzenia, jedynym rozwiązaniem jest komunikacja publiczna. Oczywistym jest jednak, że nigdy nie doprowadzimy do sytuacji, gdy wszyscy mieszkańcy zrezygnują ze swoich pojazdów. Dodatkowo, wiele usług wymaga oddzielnej formy transportu. W obliczu tych faktów wiele miast decyduje się na rozwój swojej infrastruktury drogowej. Budowa nowych tras oraz poszerzanie starych przynosi nadzieję mniejszych zatorów a co za tym idzie, szybszego przejazdu do celu. Niestety, historia pokazuje, że takie inwestycje nie zawsze przynoszą oczekiwane korzyści.\\

Teorii próbujących wytłumaczyć te zjawiska, jak również dowodów, które je popierają lub obalają jest wiele. Jedną z najpopularniejszych oraz taką która została wykorzystana w niektórych miastach na świecie jest paradoks Braessa\cite{braess}. Jest to twierdzenie matematyczne orzekające, że w pewnym modelu ruchu drogowego czasy podróży pojazdów mogą ulec wydłużeniu po dodaniu do sieci drogowej nowego połączenia. Ma ono również  zastosowanie w przypadku  sieci komputerowych oraz istnieją jego analogie dla doświadczeń fizycznych.\\

Moim celem jest opracowanie metody, która dla danej struktury sieci drogowej zmodyfikuje ją wykorzystując powyższy paradoks. W efekcie poprzez zamknięcie wybranych ulic czas podróży dla całego modelu powinien ulec skróceniu.

\section{Cele pracy}
\subsubsection{Studia literaturowe.}
Moje badania rozpocząłem od poszukiwania źródeł traktujących o opisywanym przeze mnie problemie. Paradoks Braessa został sformułowany w 1970 roku i był od tego czasu wykorzystywany przy planowaniu przestrzeni i infrastruktury wielu miast, np:

\begin{itemize}
\item Korea, Seul, likwidacja m.in. estakad Cheonggyecheon,
\item Niemcy, Stuttgart, likwidacja dróg zbudowanych w latach 60,
\item USA, Nowy Jork, czasowe zamknięcie ulicy 42,
\item USA, Winnipeg.\cite{urban}
\end{itemize}  

\subsubsection{Propozycja rozwiązania problemu.}
Oczywistym rozwiązaniem problemu komunikacji mogłoby być stworzenie idealnej sieci odpowiadającej potrzebom danego miasta. Jednak rozbudowa lub modyfikacja tej infrastruktury jest kosztowna i czasochłonna. Dlatego zdecydowałem się na przetestowanie rozwiązania zaproponowane przez Braessa. Istnieją jednak prace negujące lub podważające paradoks\cite{newinsights} , zatem przy potwierdzaniu wyniku optymalizacji nie będę kierował się wyłącznie założeniami zawartymi w twierdzeniu.

\subsubsection{Opis zastosowania algorytmów genetycznych.}
Ponieważ nie znalazłem żadnych przesłanek wykazujących jednoznaczną ocenę co do słuszności zamknięcia danej ulicy, zdecydowałem się losowe przeszukiwanie przestrzeni rozwiązań. Idealnym przykładem w przypadku takich poszukiwań są algorytmy genetyczne. 

\subsubsection{Przedstawienie oceny optymalizacji.}
Paradoks Braessa zakłada dość oczywiste potwierdzenie swojej wiarygodności. Zdecydowałem si więc na zastosowanie zewnętrznego systemu oceny. Taką rolę spełniają systemy symulacji. System, który wybrałem działa zupełnie oddzielnie od metody twierdzenia analizując zadane rozwiązanie - sieć drogową. Wynik symulacji jest jednoznaczną wartością liczbową, przedstawiającą średni czas przejazdów wszystkich agentów biorących udział w danym scenariuszu. Zakładając stały zestaw agentów dla zmieniających się w wyżej opisany sposób sieci, dążymy oczywiście do minimalizacji średniego czasu przejazdu.

\subsubsection{Ocena możliwości wdrożenia proponowanych rozwiązań.}
Paradoks Braessa nie jest jedynym twierdzeniem traktującym o problemach komunikacyjnych miast. Wiele teorii jest opartych głownie na socjologicznych lub psychologicznych założeniach. Są jednak niemniej ważne.
Biorąc pod uwagę złożoność problemu, wynik otrzymany podczas eksperymentu nie może być dowodem ani decydującym głosem w decyzjach dotyczących ustalaniu rzeczywistego ruchu drogowego miasta. 


\section{Metoda badawcza}
\begin{itemize}
\item Studia literaturowe
\item Analiza budowy i działania istniejących produktów
\item Projektowanie i prototypowanie nowatorskich rozwiązań 
\item Obliczenia i ........
\end{itemize}

Każdy element opisać w minimum 2-3 zdaniach. Np. studia literaturowe powinny
odnosić się do charakterystyki wykorzystanych źródeł książkowych, czyli: Jaka
jest podstawowa literatura dziedziny, czy jest dostępna w języku polskim, czy trzeba je tłumaczyć, czy wiedza na ten
temat jest zebrana w jednym miejscu, czy jej synteza jest osobnym zadaniem itp. 
Jak duży jest udział źródeł elektronicznych w tej ,,działce'' wiedzy i badań,
itd. \\
\indent Jakie metody badawcze są typowe dla danego tematu. Dlaczego je
zastosowano, ewentualnie dlaczego zastosowano inne? \\
WYMAGANE ODNOŚNIKI DO POZYCJI BILIOGRAFII.\\
{\bf cały podrozdział ok. 1 strony przeliczeniowej czyli 1800 znaków}.

\section{Przegląd literatury w dziedzinie}
Rozszerzyć odpowiedni podpunkt z metody badawczej, np. wg podziału:
\subsubsection{Źródła książkowe polskojęzyczne i tłumaczenia}
\subsubsection{Źródła książkowe obcojęzyczne}
\subsubsection{Artykuły naukowe, raporty z badań, komunikaty konferencyjne,
dokumentacje techniczne, manuale, instrukcje}
\subsubsection{Źródła elektroniczne}


\section{Układ pracy}
Tematem pracy jest: ....., zaś za główny cel przyjęto ...... . \\
Rozdział \label{rozdz.wstep} zawiera wstęp i cele pracy. W rozdziale drugim
opisano/...... w Rozdziale 3. zawarto............ Rozdział 4. przedstawia..... \\
W podsumowaniu pracy przedstawiono..........................., z czego wynika,
że ................  \\
Najważniejszym wnioskiem/wynikiem/rezultatem pracy jest..................\\ {\bf wyraźnie określić
CO TO JEST}. \\

{\bf cały podrozdział ok. 1 strony}.




\chapter{Optymalizacja struktury sieci drogowej} \label{etykietarozdzialu2}
\section{Podstawowe definicje}
Ten podrozdział powinien zawierać dokładny opis terminologii  pojęć zasadniczych dla tematu pracy, którymi autor będzie się posługiwał przy realizacji głównych celów pracy. 


\section{Istniejące rozwiązania w dziedzinie}
W tym podrozdziale zostaną opisane.....
\subsection{Sprzęt}
.........................
\subsection{Oprogramowanie i wdrożone systemy}
.....................................
\subsection{}
...................

\section{Wady i słabe punkty istniejących rozwiązań}
\subsection{Efektywność}
..........................
\subsection{Utrudniony dostęp}
..............
\subsection{Wysokie koszty}
...............................


\chapter[Technologie i metody użyte...]{Technologie i metody użyte}

{\em Tytuł tego rozdziału ma dwie wersje: zwykłą, (w kodzie: w nawiasach
klamrowych), która
pokazuje sie na stronie rozpoczynającej rozdział, oraz krótką (w kodzie: w nawiasach
kwadratowych), która pokazuje sie w spisie treści i w nagłówku}

W rozdziale \ref{etykietarozdzialu2} podano podstawy teoretyczne i ogólny zakres
pracy. W niniejszym rozdziale opisana zostanie technologia XYZ oraz metoda ABC
użyta w części praktycznej, patrz rozdział~\ref{rozdz.czesc.prakt}. 


\section{Sprzęt}
...................
\subsection{Element 1}
.........................



\subsection{Element 2}
......................

\section{Oprogramowanie}
..........................
\subsection{Serwer baz danych}
........................
\subsection{Środowisko zintegrowane}
..........................
\subsection{Oprogramowanie klienckie}

\section{Technologie i metodologie programistyczne}
..................
\subsection{Język programowania}
......................
\subsection{Biblioteki}
.......................
\subsection{Wzorce projektowe}
.......................

\section{Inne, np. narzędzia i metody symulacji, }

\chapter{Aplikacja/system/projekt "XYZ"} \label{rozdz.czesc.prakt}
Ta część pracy może być podzielona na więcej rozdziałów, np kiedy autor chce
w~szczególności podkreślić któryś z etapów projektu. W zależności od tematu i~celów pracy, pewne sekcje można dodać (np. przy projektowaniu sieci, instalacji
i~konfiguracji serwerów usług sieciowych), inne zaś pominąć.

\section{Analiza wymagań}
\subsection{Studium możliwości}
\subsection{Wymagania funkcjonalne}
.................
\subsection{Ograniczenia projektu}

\section{Projekt}
\subsection{Projekt warstwy danych}

\begin{enumerate}
\item normalizacje baz danych
\item projekt bazy/baz 
\item grupy użytkowników i ich prawa dostępu do danych (zależne od implementacji bazy)
\item ew. diagramy klas warstwy danych
\end{enumerate}
\subsection{Projekt warstwy logiki}
\begin{enumerate}
\item Diagramy i scenariusze przypadków użycia
\item Diagramy przepływu danych (lub ich odpowiedniki)
\item ew. diagramy klas, wzorce projektowe itp.
\end{enumerate}

\subsection{Projekt warstwy interfejsu użytkownika}
\subsubsection{Wybór środowiska i platformy działania}
\subsubsection{Rodzaj aplikacji (klient-serwer, thick/thin client, aplikacja
,,biurkowa'', usługa, klient hybrydowy, itp.}
\subsubsection{Technologie projektowania i realizacji interfejsu użytkownika,
np. biblioteki}


\section{Implementacja: punkty kluczowe}

\section{Testy i wdrożenie}
\subsection{Testy wydajności}
\subsection{Testy regresyjne}
\subsection{Testy bezpieczeństwa}
\subsection{Dalsze testy}
\subsection{Testy...}

\section{Konserwacja i inżynieria wtórna}
Jak przebiega eksploatacja systemu/projektu? Jakie wady i zalety ujawniły się po
np. 2-miesięcznym okresie testowania i użytkowania? \\
\indent Jak można skorzystać z tej wiedzy praktycznej pod kątem roz\-bu\-do\-wy pracy? Jakie elementy systemu powinny zostać w pierwszej kolejności zmodyfikowane?  

\chapter{Podsumowanie}
\section{Dyskusja wyników}
Dzięki zrealizowaniu pracy poprawie uległa wydajność ....... Ponadto, o ?? \%
skrócony został czas ........, a koszty osiągnięcia zamierzonego efektu zostały
zmniejszone z ???pln do ???pln za godzinę/ dzień/ jednostkę sprzętu.........\\
\indent Które cele pracy udało sie zrealizować? co z tego wynika? Które cele
pracy pozostały niezrealizowane i dlaczego? 

\section[Ocena możliwości wdrożenia...]{Ocena możliwości wdrożenia proponowanych
\newline rozwiązań...}
... ich wartość praktyczna, lokalne i globalne możliwości zastosowania, kwestia
praw autorskich do powstałych produktów, itp. 

\section{Perspektywy dalszych badań w dziedzinie}
Jak można kontynuować tę pracę, zwłaszcza pod kątem studiów
uzupełniających magisterskich i/lub doktoranckich. Co jeszcze powinno być
zrobione lub ulepszone? Co należy zmienić lub poprawić w pracy z dzisiejszego punktu widzenia?

\section{Definicje i wyrażenia matematyczne}
\begin{definition} \label{def.definicja1}
Niech $\cal X$ będzie przestrzenią.....
\end{definition}

\begin{lstlisting}[caption=blabla, label=amb]
def initialize(project_id)
    @project = Project.find(project_id)
end
\end{lstlisting}

\begin{table}[!t]
\centering
\caption{Tytuł tabeli ZAWSZE NAD TABELĄ, numeracja w formie \#.\#\#. (wypada podać źródło, czyli literaturę,
z której tabela pochodzi, ewentualnie {\em opracowanie własne}.)} 

\label{tabls1}

{\footnotesize 
\vspace{5mm}
\begin{tabular}{c c c c c}
\hline\noalign{\smallskip}
{\bf Alg.} & {\bf tytuł kolumny 1} & {\bf tytuł kolumny 1} & {\bf Tytuł kolumny
3} & {\bf ....}     \\

\hline\noalign{\smallskip}
a & b & c & d & e  \vspace{3mm} \\ 
\noalign{\smallskip}
 a & b & c & d & e \\

\noalign{\smallskip}
%%%
\end{tabular}
}
\end{table}

\begin{figure}[!t]
\centering
%\includegraphics[width=7cm]{1figmftall} 
\caption{Funkcja przynależności zbioru rozmytego -- Podpis ZAWSZE POD rysunkiem,
numeracja w postaci \#.\#\#. } (wypada podać źródło, czyli literaturę,
z której rysunek pochodzi, ewentualnie {\em opracowanie własne}.)
\label{fig.funkcja.przyn}
\end{figure}

\addcontentsline{toc}{chapter}{Spis rysunków} 
\listoffigures

\addcontentsline{toc}{chapter}{Spis tabel} 
\listoftables

\addcontentsline{toc}{chapter}{Spis listingów}
\lstlistoflistings

\addcontentsline{toc}{chapter}{Bibliografia} 
\begin{thebibliography}{99}
\bibitem{investigation} 
	Leslie Arthur Keith Bloy, 
	\newblock \textit{An investigation into Braess’ paradox}, 02/2007

\bibitem{newinsights}
	Rric Pas and Shari Principio
	\newblock \textit{Braess’ paradox: Some new insights}, April 1996

\bibitem{conference} 
	Wataru Nanya, Hiroshi Kitada, Azusa Hara, Yukiko Wakita, Tatsuhiro Tamaki, and Eisuke Kita
	\newblock \textit{Road Network Optimization for Increasing Traffic Flow}
	\newblock Int. Conference on Simulation Technology, JSST 2013.

\bibitem{reducingtheeffects}
	Ana L. C. Bazzan and Franziska Klügl
	\newblock \textit{Reducing the Effects of the Braess Paradox with Information Manipulation}

\bibitem{matsim} 
	\url{http://matsim.org}	

\bibitem{math}
	\url{http://commons.apache.org/proper/commons-math}

\bibitem{java} 
	\url{http://www.java.com/pl/}

\bibitem{eclipse} 
	\url{https://eclipse.org}
				
\bibitem{python} 
	\url{http://pl.python.org}
	
\bibitem{pydev} 
	\url{http://pydev.org}
	
\bibitem{trisquel}
	\url{https://trisquel.info}
			
\bibitem{matsim-userg}
	M. Rieser, C. Dobler, T. Dubernet, D. Grether, A. Horni, G. Lammel, R. Waraich, M. Zilske, Kay W. Axhausen, Kai Nagel
	\newblock \textit{MATSim User Guide}
	\newblock updated September 12, 2014

\bibitem{siux}
	A. Chakirov
	\newblock \textit{Enriched Sioux Falls Scenario with Dynamic Demand}
	\newblock MATSim User Meeting, Zurich/Singapore, June 2013.
	
\bibitem{braess}
	\url{http://pl.wikipedia.org/wiki/Paradoks_Braessa}
	
\bibitem{urban}
	\url{http://urbnews.pl/paradoks-braessa/}

\end{thebibliography}


\addcontentsline{toc}{chapter}{Załączniki} 
\chapter*{Załączniki}
\begin{enumerate}
\item Załącznik nr 1
\item Załącznik nr 2
\item Załącznik nr 3
\end{enumerate}

\chapter*{Abstract}

The purpose of the present bachelor thesis was to create an internet application with an integrated recommender system based on music resources. My work covered two main fields: creating the application as well as building a~recommender system and testing it efficiency.\\



\end{document}
